% !TeX spellcheck = sk_SK-Slovak
\documentclass[a4paper]{article}
\usepackage[slovak]{babel}
\usepackage[utf8]{inputenc}
\usepackage[T1]{fontenc}
\usepackage{a4wide}
\usepackage{amsmath}
\usepackage{amsfonts}
\usepackage{amssymb}
\usepackage{mathrsfs}
\usepackage[small,bf]{caption}
\usepackage{subcaption}
\usepackage{xcolor}
\usepackage{graphicx}
\usepackage{enumerate}
\usepackage{hyperref}
\usepackage{fancyvrb}
\usepackage{listings}
%\usepackage{lstautogobble}
\usepackage{stmaryrd}

\lstset{basicstyle=\ttfamily,
	mathescape=true,
	escapeinside=||%,
	%autogobble
}


\fvset{tabsize=4}


\pagestyle{empty}
\setlength{\parindent}{0pt}

\newenvironment{modenumerate}
{\enumerate\setupmodenumerate}
{\endenumerate}

\newif\ifmoditem
\newcommand{\setupmodenumerate}{%
	\global\moditemfalse
	\let\origmakelabel\makelabel
	\def\moditem##1{\global\moditemtrue\def\mesymbol{##1}\item}%
	\def\makelabel##1{%
		\origmakelabel{##1\ifmoditem\rlap{\mesymbol}\fi\enspace}%
		\global\moditemfalse}%
}

\makeatletter
\def\@seccntformat#1{%
	\expandafter\ifx\csname c@#1\endcsname\c@section\else
	\csname the#1\endcsname\quad
	\fi}
\makeatother

\begin{document} 
	
\pagenumbering{arabic}
\pagestyle{plain}

\begin{center}
	\sc\large
	Algoritmické riešenia ťažkých problémov\\
	Skúška
\end{center}

\subsection*{Úloha 1 - Výber správnej odpovede}

a) Určite správnu odpoveď o pravdivosti tvrdení:

-> Las Vegas algoritmus nikdy nebude trvať nekonečne dlho.

-> Pravdepodobnosť, že Las Vegas algoritmus bude trvať nekonečne dlho je nulová

Výber: prvé platí, druhé platí, ani jedno neplatí
\\

b) Máme minimalizačný problém v podobe ILP ktorý má optimálne riešenie $OPT(X)$, ak tento problém relaxujeme do podoby LP (nazvime RLP), nájdeme optimum LP a vypočítame riešenie pôvodného problému metódou zaokrúhľovania, čo vieme povedať o výsledku $RLP(X)$:

Výber: $RLP(X) \leq OPT(X)$, $RLP(X) \geq OPT(X)$, $RLP(X) = OPT(X)$, nevieme určiť
\\

c)
\\

d) Máme štandardný random walk, chceme sa dostač na pozíciu 100, začíname na pozícii 34, aký je očakávaný počet krokov? (zdôvodnite)

Výber: 0, 128, 8844, iná hodnota
\\

e) 

\subsection*{Úloha 2 - Z LV do MC}

Máme Las Vegas algoritmus s očakávaným časom $E(LV)=n$. Chceme vytvoriť Monte Carlo algoritmus pomocou tohto algoritmu.
\\

a) Aká je pravdepodobnosť, že Las Vegas algoritmus skončí za menej ako $10n$ krokov?
\\

b) Napíšte pseudokód Monte Carlo algoritmu využívajúceho Las Vegas algorimus.
\\

c) Čo musíme spraviť aby chyba Monte Carlo algoritmu bola menšia ako $1\%$?

\subsection*{Úloha 3 - Medianové slovo}

Máme množinu m slov dĺžky n, slovo je postupnosť znakov z množiny $\{a,b\}$ (príklad: $\{abba, baba, bbbb\}$). Vzdialenosť dvoch slov definujeme ako počet pozícii na ktorých sa slová nezhodujú (príklady: $d(aba, aaa) = 1$, $d(aba, bab) = 3$). Hľadáme medianové slovo, čiže slovo s najmenšou vzdialenosťou od najvzdialenejšieho slova $x_{med} = \min_y(\max(\forall x_i, d(y, x_i)))$. Profesor Premúdretý tvrdí, že medianové slovo musí patriť do množiny slov, takže stačí skontrolovať vzájomné vzdialenosti všetkých slov z množiny a nájsť najlepšie.
\\

a) Ukážte, že algoritmus prof. Premúdretého nedá vždy optimálne riešenie.
\\

b) Dokážte, že platí trojuholníkové pravidlo $d(x,z) \leq d(x,y) + d(y,z)$.
\\

c) Dokážte, že algoritmus prof. Premúdretého je 2-APX (môžte využiť pravidlo z b) aj keď ste ho nedokázali).

\subsection*{Úloha 4 - Parametrizovaný IS}

Chceme riešiť problém nezávislej množiny, pričom poznáme $k$ čo je veľkosť nezávislej množiny a $\Delta$ čo je maximálny stupeň vrchola. 
\\

a) Dokážte, že platí tvrdenie: $\forall v \in V, v \in IS \lor \exists neig(v) \in IS $, pre každý vrchol v grafe platí, že buď vrchol patrí do nezávislej množiny alebo aspoň jeden jeho sused tam patrí.
\\

b) Vymyslite polynomialny algoritmus parametrizovaný $k$ a $\Delta$ ktorý rieši problém nezávislej množina (môžte využiť tvrdenie z a) aj keď ste ho nedokázali).
\\

c) Aká je časová zložitosť tohto algoritmu?
\\

d) Ak máme určené iba $k$ (nemáme vopred dané $\Delta$) bude tento algoritmus stále polynomialny? (svoje tvrdenie dokážte)

\subsection*{Úloha 5 - BONUS - Fibonacci random walk}

Máme random walk kde  pozície $i$ sa posunieme s pravdepodobnosťou s rovnakou na pozíciu $i-1$ a $i+2$ (okrem pozícii $n$ a $n-1$ kde vieme ísť iba na pozície $i-1$), chceme sa dostať na pozíciu 0, vypočítajte očakávaný počet krokov (a ukážte, že rastie exponenciálne).

\end{document}