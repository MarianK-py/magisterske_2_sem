% !TeX spellcheck = sk_SK-Slovak
\documentclass[a4paper]{article}
\usepackage[slovak]{babel}
\usepackage[utf8]{inputenc}
\usepackage[T1]{fontenc}
\usepackage{a4wide}
\usepackage{amsmath}
\usepackage{amsfonts}
\usepackage{amssymb}
\usepackage{mathrsfs}
\usepackage[small,bf]{caption}
\usepackage{subcaption}
\usepackage{xcolor}
\usepackage{graphicx}
\usepackage{enumerate}
\usepackage{hyperref}
\usepackage{fancyvrb}
\usepackage{listings}
%\usepackage{lstautogobble}
\usepackage{stmaryrd}

\lstset{basicstyle=\ttfamily,
	mathescape=true,
	escapeinside=||%,
	%autogobble
}


\fvset{tabsize=4}


\pagestyle{empty}
\setlength{\parindent}{0pt}

\newenvironment{modenumerate}
{\enumerate\setupmodenumerate}
{\endenumerate}

\newif\ifmoditem
\newcommand{\setupmodenumerate}{%
	\global\moditemfalse
	\let\origmakelabel\makelabel
	\def\moditem##1{\global\moditemtrue\def\mesymbol{##1}\item}%
	\def\makelabel##1{%
		\origmakelabel{##1\ifmoditem\rlap{\mesymbol}\fi\enspace}%
		\global\moditemfalse}%
}

\makeatletter
\def\@seccntformat#1{%
	\expandafter\ifx\csname c@#1\endcsname\c@section\else
	\csname the#1\endcsname\quad
	\fi}
\makeatother

\begin{document} 
	
\pagenumbering{arabic}
\pagestyle{plain}

\begin{center}
	\sc\large
	Algoritmické riešenia ťažkých problémov\\
	Domáca úloha 3
\end{center}

Autor: Marián Kravec

\section{Úloha 2 - A random walk on a tree}

Chceme dokázať, že pre náš algoritmus platí $E(ALG(n)) \leq log_2(n+1)$.
\\

Budeme to dokazovať pomocou matematickej indukcie.

Začnime bázovým príkladom, máme $n=3$ vrcholy (jeden vnútorný a jeho dva listy).
Pre tento strom platí, že do každého listu sa dostane na jeden krok z koreňu takže $E(ALG(3)) = 1 \le 2 = log_2(3+1)$. (triviálny prípad by mohol byť aj strom s jedným vrcholom kde by platilo, že algoritmus spraví 0 krokov čiže takisto spĺňa $E(ALG(1)) = 0 = log_2(0+1)$, pre $n=2$ neexistuje strom ktorý by spĺňal, že všetky vnútorné vrcholy majú dvoch potomkov preto túto hodnotu neriešime)

Ako indukčný predpoklad uvažujme, že toto tvrdenie platí, pre všetky $k$ menšie ako $n$.

Teraz chceme ukázať indukčný krok čiže, že to platí aj pre $n$. Keďže začíname v koreni čo je vnútorný vrchol, existujú z neho cesty do dvoch podstromov. Pričom prvý krok algoritmu prejde do jedného z týchto podstromov. Uvažujme, že ľavý podstrom má $n_1$ vrcholov a pravý má $n_2$ vrcholov, z toho vieme vyvodiť rovnice $n=n_1+n_2+1$ (celkový počet vrcholov je súčet vrcholov podstromov plus koreň), $n_1<n$, $n_2<n$ (podstromy musia mať manej vrcholov ako celý strom keďže minimálne neobsahujú koreň). Keďže vieme, že pravdepodobnosť kroku do každého z podstromov je rovnaká a zvyšný počet krokov je závislý od veľkosti podstromu platí takáto rovnica:

$E(ALG(n)) = \frac{1}{2}E(ALG(n_1)) + \frac{1}{2}E(ALG(n_2)) + 1$ (čiže pravdepodobnosť, že pôjdem do daného podstromu vynásobená očakávyným počtom krokov v danom podstrome plus nakoniec ten jeden krok do podstromu)

Keďže platí $n_1<n$, $n_2<n$ a vieme, že naše tvrdenie platí pre všetky $k$ menšie ako $n$ tak platí $E(ALG(n_1)) \leq log_2(n_1+1)$, $E(ALG(n_2)) \leq log_2(n_2+1)$.

Ak toto dosadíme do predchádzajúcej rovnice dostaneme:

$E(ALG(n)) = \frac{1}{2}E(ALG(n_1)) + \frac{1}{2}E(ALG(n_2)) + 1 \leq \frac{1}{2}log_2(n_1+1) + \frac{1}{2}log_2(n_2+1) + 1$

$E(ALG(n)) \leq \frac{1}{2}(log_2(n_1+1) + log_2(n_2+1)) + 1$

Vieme, že súčet logaritmov je logaritmus súčinu:

$E(ALG(n)) \leq \frac{1}{2}log_2((n_1+1)(n_2+1)) + 1$

Takisto vieme, že platí $n*log(x)=log(x^n)$:

$E(ALG(n)) \leq log_2(((n_1+1)(n_2+1))^{\frac{1}{2}}) + 1$

Ďalej vieme, že $x^{\frac{1}{2}} = \sqrt{x}$:

$E(ALG(n)) \leq log_2(\sqrt{(n_1+1)(n_2+1)}) + 1$

Teraz využijeme vlastnosť, že logaritmus je monotónne rastúca funkcia (čiže platí, že $a\leq b$ tak $log(a) \leq log(b)$) a platí AG nerovnosť $\sqrt[n]{x_1*x_2*...*x_n} \leq \frac{x_1+x_2+...+x_n}{n}$. Ak toto substituujeme v našej rovnici dostaneme:

$E(ALG(n)) \leq log_2(\sqrt{(n_1+1)(n_2+1)}) + 1 \leq log_2(\frac{(n_1+1)+(n_2+1)}{2}) + 1$ 

Teraz si rozpíšme sumu v menovateli:

$E(ALG(n)) \leq log_2(\frac{(n_1+n_2+1+1)}{2}) + 1$ 

Vieme, že platí $n=n_1+n_2+1$:

$E(ALG(n)) \leq log_2(\frac{(n+1)}{2}) + 1$ 

Ďalej vieme, že logaritmus podielu je rozdiel logaritmov:

$E(ALG(n)) \leq log_2(n+1) - log_2(2) + 1$ 

Vieme, že $log_2(2) = 1$

$E(ALG(n)) \leq log_2(n+1) - 1 + 1 = log_2(n+1)$ 

Takže sme dostali výsledok:

$E(ALG(n)) \leq  log_2(n+1)$ 

Čo je to čo sme chceli dokázať. $\oblong$

\end{document}