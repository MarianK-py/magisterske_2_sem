% !TeX spellcheck = sk_SK-Slovak
\documentclass[a4paper]{article}
\usepackage[slovak]{babel}
\usepackage[utf8]{inputenc}
\usepackage[T1]{fontenc}
\usepackage{a4wide}
\usepackage{amsmath}
\usepackage{amsfonts}
\usepackage{amssymb}
\usepackage{mathrsfs}
\usepackage[small,bf]{caption}
\usepackage{subcaption}
\usepackage{xcolor}
\usepackage{graphicx}
\usepackage{enumerate}
\usepackage{hyperref}
\usepackage{fancyvrb}
\usepackage{listings}
%\usepackage{lstautogobble}
\usepackage{stmaryrd}

\lstset{basicstyle=\ttfamily,
	mathescape=true,
	escapeinside=||%,
	%autogobble
}


\fvset{tabsize=4}


\pagestyle{empty}
\setlength{\parindent}{0pt}

\newenvironment{modenumerate}
{\enumerate\setupmodenumerate}
{\endenumerate}

\newif\ifmoditem
\newcommand{\setupmodenumerate}{%
	\global\moditemfalse
	\let\origmakelabel\makelabel
	\def\moditem##1{\global\moditemtrue\def\mesymbol{##1}\item}%
	\def\makelabel##1{%
		\origmakelabel{##1\ifmoditem\rlap{\mesymbol}\fi\enspace}%
		\global\moditemfalse}%
}

\makeatletter
\def\@seccntformat#1{%
	\expandafter\ifx\csname c@#1\endcsname\c@section\else
	\csname the#1\endcsname\quad
	\fi}
\makeatother

\begin{document} 
	
\pagenumbering{arabic}
\pagestyle{plain}

\begin{center}
	\sc\large
	Algoritmické riešenia ťažkých problémov\\
	Domáca úloha 3
\end{center}

Autor: Marián Kravec

\section{Úloha 1 -  Random permutations}

\subsection*{a)}

Ukážeme to induktívne.
\\

Najprv ako bázu indukcie budeme mať zoznam dĺžky 1. Pre takýto zoznam existuje iba jedna permutácia čo je samotný pôvodný zoznam a keďže náš algoritmus začína prehadzovať od druhej pozície na tomto zozname nikdy nič nezmení, takže spĺňa, že vytvorí všetky permutácie (jednu) s rovnakou pravdepodobnosťou (keďže obe pozície s ktorými vymenil ).

Pre istotu ešte ako bázu použime zoznam dĺžky 2. V tomto zozname spraví náš algoritmus iba jednu zmenu a to, že druhý člen zoznamu vymení s buď sám sebou alebo prvým (obe s rovnakou pravdepodobnosťou) čím vytvorí $2$ permutácie čo je $2!$ čím spĺňa, že vytvorí všetky permutácie s rovnakou pravdepodobnosťou.
\\

Teraz uvažujme, že náš vlastnosti nášho algoritmu platia pre všetky zoznamy dĺžky $k<n$.
\\

Chceme ukázať, že platia aj pre zoznam dĺžka $n$. Vieme, že po $n-1$ krokoch náš algoritmus vytvorí jednu z $(n-1)!$ permutácii s rovnakou pravdepodobnosťou (indukčný predpoklad) ktorú označíme $p_{n-1}=\frac{1}{(n-1)!}$. V poslednom kroku vymení posledný člen s niektorým z $n$ členov (všetky výmeny majú rovnakú pravdepodobnosťou). Čiže nakoniec dostaneme jeden z $n!$ zoznamov kde každý vznikol s pravdepodobnosťou  $p_{n-1}\frac{1}{n}=\frac{1}{(n-1)!}\frac{1}{n}=\frac{1}{n!}$ (keďže doterajšie pravdepodobnosti boli všetky rovnaké a pravdepodobnosť poslednej výmeny je rovnomerná). Teraz už iba potrebujeme ukázať, že tento posledný krok nemôže vytvoriť dva rovnaké zoznamy. To dokážeme sporom.
\\

Uvažujem, že po poslednom kroku máme 2 zoznamy $A=[a_1,...a_n]$ a $B=[b_1,...b_n]$ ktoré sú rovnaké, čiže platí $\forall 1\leq i \leq n, a_i = b_i$, ale pred týmto krokom to boli dve rôzne permutácie $(n-1)$ prvkov. 

Bez ujmy na všeobecnosti uvažujme, že v poslednom kroku vymenil algoritmus $a_i$ s $a_n$ a $b_j$ s $b_n$, keďže na konci bola na n-tej pozícii tá istá hodnota muselo pred výmenou platiť $a_i = b_j$ a keďže na konci bola hodnota $n$ na rovnakej pozícii (vždy je inicializovaná na n-tej pozícii a predchádzajúce kroky ju nevedia posunúť) tak musí platiť $i=j$, takže pred poslednou výmenou platilo $a_i = b_i$ a keďže okrem tejto pozície posledný krok nezmenil žiadnu inú a na konci platilo  $\forall 1\leq k \leq n, a_k = b_k$. Tak pred týmto krokom muselo platiť  $\forall 1\leq k \leq n-1, k \neq i, a_k = b_k$ a keďže platí aj $a_i = b_i$ tak pred posledným krokom boli zoznamy $A$ a $B$ rovnaké čo je v spore s predpokladom, že ide o dve rôzne permutácie.
\\

Čiže sme ukázali, že po poslednom kroku budeme mať jednu z $n!$ rôznych permutácii každú s rovnakou pravdepodobnosťou, čo je to čo sme chceli ukázať. $\oblong$


  
 
\end{document}