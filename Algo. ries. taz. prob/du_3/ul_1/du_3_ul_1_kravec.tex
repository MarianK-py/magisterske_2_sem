% !TeX spellcheck = sk_SK-Slovak
\documentclass[a4paper]{article}
\usepackage[slovak]{babel}
\usepackage[utf8]{inputenc}
\usepackage[T1]{fontenc}
\usepackage{a4wide}
\usepackage{amsmath}
\usepackage{amsfonts}
\usepackage{amssymb}
\usepackage{mathrsfs}
\usepackage[small,bf]{caption}
\usepackage{subcaption}
\usepackage{xcolor}
\usepackage{graphicx}
\usepackage{enumerate}
\usepackage{hyperref}
\usepackage{fancyvrb}
\usepackage{listings}
%\usepackage{lstautogobble}
\usepackage{stmaryrd}

\lstset{basicstyle=\ttfamily,
	mathescape=true,
	escapeinside=||%,
	%autogobble
}


\fvset{tabsize=4}


\pagestyle{empty}
\setlength{\parindent}{0pt}

\newenvironment{modenumerate}
{\enumerate\setupmodenumerate}
{\endenumerate}

\newif\ifmoditem
\newcommand{\setupmodenumerate}{%
	\global\moditemfalse
	\let\origmakelabel\makelabel
	\def\moditem##1{\global\moditemtrue\def\mesymbol{##1}\item}%
	\def\makelabel##1{%
		\origmakelabel{##1\ifmoditem\rlap{\mesymbol}\fi\enspace}%
		\global\moditemfalse}%
}

\makeatletter
\def\@seccntformat#1{%
	\expandafter\ifx\csname c@#1\endcsname\c@section\else
	\csname the#1\endcsname\quad
	\fi}
\makeatother

\begin{document} 
	
\pagenumbering{arabic}
\pagestyle{plain}

\begin{center}
	\sc\large
	Algoritmické riešenia ťažkých problémov\\
	Domáca úloha 3
\end{center}

Autor: Marián Kravec

\section{Úloha 1 -  Random permutations}

\subsection*{a)}

Začnime tým, že na konci procesu generovania náhodnej permutácie máme zoznam $S=[a_1,a_2,...,a_{n-1}, a_n]$. Teraz sa pozrime na posledný krok pri jej generovaní, hodnotu $S[n]$ sme vymenili z hodnotou $S[rand(1, n)]$, keďže vieme, že na $n$-tej pozícii bolo nakoniec $a_n$ tak musí platiť $rand(1, n)=u$ pričom platilo, že pred týmto krokom $S[u]=a_n$, keďže vieme, že každá hodnota je v zozname práve raz tak platí $P(rand(1, n)=u) = \frac{1}{n}$ (keďže funkcia $rand$ vráti hodnoty rovnomerne). 
  
 
\end{document}