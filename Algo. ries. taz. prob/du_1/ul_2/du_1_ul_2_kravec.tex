% !TeX spellcheck = sk_SK-Slovak
\documentclass[a4paper]{article}
\usepackage[slovak]{babel}
\usepackage[utf8]{inputenc}
\usepackage[T1]{fontenc}
\usepackage{a4wide}
\usepackage{amsmath}
\usepackage{amsfonts}
\usepackage{amssymb}
\usepackage{mathrsfs}
\usepackage[small,bf]{caption}
\usepackage{subcaption}
\usepackage{xcolor}
\usepackage{graphicx}
\usepackage{enumerate}
\usepackage{hyperref}
\usepackage{fancyvrb}
\usepackage{listings}
%\usepackage{lstautogobble}
\usepackage{stmaryrd}

\lstset{basicstyle=\ttfamily,
	mathescape=true,
	escapeinside=||%,
	%autogobble
}


\fvset{tabsize=4}


\pagestyle{empty}
\setlength{\parindent}{0pt}

\newenvironment{modenumerate}
{\enumerate\setupmodenumerate}
{\endenumerate}

\newif\ifmoditem
\newcommand{\setupmodenumerate}{%
	\global\moditemfalse
	\let\origmakelabel\makelabel
	\def\moditem##1{\global\moditemtrue\def\mesymbol{##1}\item}%
	\def\makelabel##1{%
		\origmakelabel{##1\ifmoditem\rlap{\mesymbol}\fi\enspace}%
		\global\moditemfalse}%
}

\makeatletter
\def\@seccntformat#1{%
	\expandafter\ifx\csname c@#1\endcsname\c@section\else
	\csname the#1\endcsname\quad
	\fi}
\makeatother

\begin{document} 
	
\pagenumbering{arabic}
\pagestyle{plain}

\begin{center}
	\sc\large
	Algoritmické riešenia ťažkých problémov\\
	Domáca úloha 1
\end{center}

Autor: Marián Kravec

\section{Úloha 2 - Bureaucrat}

Na začiatok predpoklad: žiadna úloha nemá zápornú dĺžku ($\forall a_i: a_i\geq 0$)

\subsection*{a)}

Najskôr si zadefinujme ľahko vypočítateľné dolné ohraničenie optimálneho riešenia. Máme $k$ byrokratov a $n$ úloh dĺžok $a_1, ..., a_n$, čiže celkový súčet úloh je $A = \sum_{i=1}^{n} a_i$. Bez ohľadu na to ako tieto úlohy rozdelíme tak v priemere bude mať byrokrat celkovú dĺžku úloh $\frac{A}{k}$. Keďže ide o priemer vieme, že maximum nemôže byť menšie ako tento priemer, takže pre každé riešenie platí, že musí existovať byrokrat ktorého celková dĺžka úloh bude aspoň priemerná dĺžka, z čoho dostávame dolné ohraničenie $OPT(X) \geq \frac{A}{k}$.

Teraz si ukážeme príklad kedy daný algoritmus nevráti optimálne rozdelenie úloh medzi byrokratov. Majme dvoch byrokratov a 6 úloh dĺžok 7, 5, 3, 3, 3, 3.
Optimálne riešenie je v tomto prípade dať jednému byrokratovi úlohy dĺžky 7 a 5 (v súčte 12) a druhému všetky úlohy dĺžky 3 (v súčte 12), keďže celkový súčet úloh je 24 a máme dvoch byrokratov tak vieme riešenia zdola ohraničiť hodnotou 12 ($\frac{24}{2}$), naše riešenie túto hodnotu nadobúda preto o ňom môžeme povedať, že je optimálne. Teraz sa pozrime, čo spraví náš algoritmus: 

\begin{table}[!h]
	\begin{tabular}{|p{0.08\textwidth}|p{0.095\textwidth}|p{0.095\textwidth}|p{0.125\textwidth}|p{0.095\textwidth}|p{0.095\textwidth}|}
		\hline
		dĺžka úlohy & byrokrat 1 pred & byrokrat 2 pred & kto dostane úlohu & byrokrat 1 po & byrokrat 2 po \\ \hline
		7           & 0               & 0               & byrokrat 1        & 7             & 0             \\ \hline
		5           & 7               & 0               & byrokrat 2        & 7             & 5             \\ \hline
		3           & 7               & 5               & byrokrat 2        & 7             & 8             \\ \hline
		3           & 7               & 8               & byrokrat 1        & 10            & 8             \\ \hline
		3           & 10              & 8               & byrokrat 2        & 10            & 11            \\ \hline
		3           & 10              & 11              & byrokrat 1        & 13            & 11            \\ \hline
	\end{tabular}
\end{table}
Vidíme, že na konci algoritmu má byrokrat 1 celkovú sumu 13 a byrokrat 2 má sumu 11. Maximum z týchto súm je 13, čo je hodnota nášho riešenia, táto hodnota je vyššia ako hodnota optimálneho riešenia 12, čiže toto riešenie je suboptimálne.
\\

%Teraz sa pozrime, ako bude vyzerať riešenie nášho algoritmu pre vstup v zadaní:
%\begin{table}[!h]
	%\begin{tabular}{|p{0.08\textwidth}|p{0.095\textwidth}|p{0.095\textwidth}|p{0.095\textwidth}|p{0.125\textwidth}|p{0.095\textwidth}|p{0.095\textwidth}|p{0.095\textwidth}|}
		%\hline
		%dĺžka úlohy & byrokrat 1 pred & byrokrat 2 pred & byrokrat 3 pred & kto dostane úlohu & byrokrat 1 po & byrokrat 2 po & byrokrat 3 po \\ \hline
		%14          & 0               & 0               & 0               & byrokrat 1        & 14            & 0             & 0             \\ \hline
		%10          & 14              & 0               & 0               & byrokrat 2        & 14            & 10            & 0             \\ \hline
		%7           & 14              & 10              & 0               & byrokrat 3        & 14            & 10            & 7             \\ \hline
		%4           & 14              & 10              & 7               & byrokrat 3        & 14            & 10            & 11            \\ \hline
		%2           & 14              & 10              & 11              & byrokrat 2        & 14            & 12            & 11            \\ \hline
		%2           & 14              & 12              & 11              & byrokrat 3        & 14            & 12            & 13            \\ \hline
		%1           & 14              & 12              & 13              & byrokrat 2        & 14            & 13            & 13            \\ \hline
%	\end{tabular}
%\end{table}

%Vidíme, že náš algoritmus dá riešenie 14 čo je optimálne riešenie. To že je to optimálne môžeme ukázať znovu použitím priemeru dĺžky úloh na byrokrata. V tomto prípade je súčet všetkých úloh 40, keďže máme troch byrokratov, priemer je $13.\bar{3}$ avšak keďže sú v tomto prípade všetky úlohy celočíselnej dĺžky aj optimum musí byť celočíselné preto optimum nemôže byť menšie ako 14, čo je výsledok nášho algoritmu. 

\subsection*{b)}

Chceme dokázať, že pre všetky $a_i$ platí $a_i > \frac{t^*}{3}$ kde $t^*$ je hodnota optimálneho riešenia, tak algoritmus vráti optimálne riešenie.
\\

Najskôr si dokážme takéto tvrdenie ak platí $\forall a_i: a_i > \frac{t^*}{3}$ tak žiaden byrokrat nedostane viac ako 2 úlohy. (nulové úlohy ignorujeme)
\\

Toto dokážeme sporom, čiže máme tvrdenie, že ak $\forall a_i: a_i > \frac{t^*}{3}$ tak môže existovať byrokrat ktorý dostane aspoň tri úlohy.
Uvažujme, že máme byrokrata $b$ ($b$ označuje celkovú dĺžku práce byrokrata) ktorý má aspoň tri úlohy, medzi je úlohami sú úlohy $i$, $j$ a $k$, čiže dĺžka je práce je aspoň súčet dĺžok týchto úloh $b \geq a_i + a_j + a_k$ avšak o každej z týchto úloh vieme povedať $a_i > \frac{t^*}{3}$ preto platí $b \geq a_i + a_j + a_k > \frac{t^*}{3} + \frac{t^*}{3} +\frac{t^*}{3} = t^*$, čiže $b > t^*$ čo je v spore z predpokladom, že $t^*$ je optimálne riešenie, čiže neexistuje byrokrat ktorý má súčet dĺžok úloh väčší ako je táto hodnota. Preto platí, že každý byrokrat má nanajvýš 2 úlohy.
\\

Teraz sa pozrime, čo spraví náš algoritmus. Bez ujmy na všeobecnosti si našich byrokratov nejak usporiadajme. Náš algoritmus najskôr postupne zľava doprava každému byrokratovi jednu najväčšiu zatiaľ nepridanú úlohy kým nemá každý práve jednu úlohy alebo nie sú pridané všetky úlohy(vieme to zaručiť lebo kým nepridá prvých $k$ úloh existuje byrokrat s celkovou sumou 0 čo je najnižšia možná). Ďalej uvažujme o prípade, že každý byrokrat má pridelenú jednu úlohu a ešte existujú nepridelené úlohy, keďže sme pridávali úlohy zľava doprava od najväčšej sme teraz v situácii, že sú byrokrati zoradený podľa celkových dĺžok ich úloh. Preto keď algoritmus začne prideľovať zvyšné úlohy, bude ich prideľovať z sprava doľava po jednej úlohy (vieme, že iba po jednej lebo po pridelení ďalšej úlohy bude mať byrokrat 2 úlohy čo sme dokázali, že za daných podmienok je maximum). 
\\

Teraz potrebujeme ukázať, že toto riešenie je optimálne. Na to si ukážeme niekoľko vlastností z ktorých optimalita riešenia vyplynie.

\begin{itemize}
	\item Každá úloha má menšiu nanajvýš rovnú hodnotu ako optimum ($\forall a_i, a_i \leq t^*$). Toto je triviálne keďže potrebujem prideliť všetky úlohy, ak by existovala úlohy s hodnotou väčšou ako optimum musela by byť pridelená, čo by však znamenalo, že celková hodnota úloh pre byrokrata prekračuje hodnotu optimálne riešenia čo vytvára spor.
	
	\item Algoritmus nikdy nepridelí byrokratovi druhú úlohu ak jeho prvá má hodnotu väčšiu, nanajvýš rovnú ako $\frac{2}{3}$ optimálneho riešenia ($a_i \geq \frac{2}{3}t^*$). Dokážeme to sporom. Uvažujme, že náš algoritmus pridelí takému byrokratovi druhú úlohu, to by znamenalo, že všetci byrokrati ktorých prvá úloha mala hodnotu menšiu ako $\frac{2}{3}t^*$ už svoju druhú úlohu dostali. Z toho ale vyplýva, že aj v optimálnom riešení musí byť byrokrat s dvomi úlohami kde jedna má hodnotu väčšiu, nanajvýš rovnú ako $\frac{2}{3}t^*$ (vyplýva to z Dirichletovho princípu, keďže po prvom pridelovaní úloh nám ostane viac úloh ako byrokratov s úlohou hodnoty menšej ako $\frac{2}{3}t^*$) to je však spor, keďže v takom prípade by v optimálnom riešení bol byrokrat ktorý má prvú úlohu $a_i \geq \frac{2}{3}t^*$ a druhú $a_j > \frac{1}{3}t^*$ čo je v súčte $a_i + a_j \geq \frac{2}{3}t^* + \frac{1}{3}t^* = t^*$ čo je v spore s tým, že $t^*$ je optimálne riešenie.
	
	\item Ak náš algoritmus pridelí 2 úlohy byrokratovi ich súčet bude menší nanajvýš rovný ako optimálne riešenie ($a_i + a_j \leq t^*$).
	Už vieme, že každá obe úlohy sú menšie ako $\frac{2}{3}t^*$, z čoho vyplýva, že ich súčet je menší ako  $\frac{4}{3}t^*$. To, že v skutočnosti táto hodnota hodnotu optima neprekročí ukážeme sporom. Uvažujme, že v našom riešení existuje byrokrat ktorého súčet úloh prekračuje optimum. Uvažujme, že ide o byrokrata s maximálnou celkovou hodnotou úloh (čiže výsledok algoritmu). V takom prípade nastala situácia keď náš algoritmus mu prideľoval druhú úlohu v tej chvíli všetci naľavo od neho majú jednú úlohu s hodnotou väčšou ako je jeho prvá a všetci napravo majú dve úlohy, ak by sa práve prideľovaná úloha pridelila niekomu naľavo od tohto byrokrata bol by celkový súčet úloh pre daného byrokrata naľavo rovný alebo ešte väčší ak by bola úloha pridelená podľa algoritmu, čiže by sa nezmenšila hodnota riešenia, podobne ak by túto úlohu algoritmus chcel prideliť niektorému z byrokratov napravo musel by danému byrokratovi jednu z jemu už pridelených úloh odobrať (keďže sme ukázali, že žiaden byrokrat nesmie mať viac ako 2 úlohy), avšak všetky úlohy pridelené napravo sú väčšie nanajvýš rovné práve prideľovanej úlohe (keďže ich prideľujeme od najväčšej) z toho vyplýva, že ak by algoritmus chcel prideliť práve prideľovanú úlohu niekomu napravo musel by v ďalšom kroku prideľovať úlohu z väčšou hodnotou čo by znovu nemohlo zmenšiť hodnotu riešenia. Z toho vyplýva, že všetky riešenia majú byrokrata s súčtom väčším ako optimálne riešenie čo je ale v spore s tým, že optimálne riešenie túto hodnotu nadobúda.  
\end{itemize}

Z týchto čiastkových dôkazov nám vyplýva, že bez ohľadu na to či byrokrat dostane jednu alebo dve úlohy ich súčet nepresiahne optimálne riešenie, a keďže sa optimálne riešenie nedá podliezť tak ho náš algoritmus dosiahne.

\subsection*{c)}

%I have no FUCKING idea


\end{document}