% !TeX spellcheck = sk_SK-Slovak
\documentclass[a4paper]{article}
\usepackage[slovak]{babel}
\usepackage[utf8]{inputenc}
\usepackage[T1]{fontenc}
\usepackage{a4wide}
\usepackage{amsmath}
\usepackage{amsfonts}
\usepackage{amssymb}
\usepackage{mathrsfs}
\usepackage[small,bf]{caption}
\usepackage{subcaption}
\usepackage{xcolor}
\usepackage{graphicx}
\usepackage{enumerate}
\usepackage{hyperref}
\usepackage{fancyvrb}
\usepackage{listings}
%\usepackage{lstautogobble}
\usepackage{stmaryrd}

\lstset{basicstyle=\ttfamily,
	mathescape=true,
	escapeinside=||%,
	%autogobble
}


\fvset{tabsize=4}


\pagestyle{empty}
\setlength{\parindent}{0pt}

\newenvironment{modenumerate}
{\enumerate\setupmodenumerate}
{\endenumerate}

\newif\ifmoditem
\newcommand{\setupmodenumerate}{%
	\global\moditemfalse
	\let\origmakelabel\makelabel
	\def\moditem##1{\global\moditemtrue\def\mesymbol{##1}\item}%
	\def\makelabel##1{%
		\origmakelabel{##1\ifmoditem\rlap{\mesymbol}\fi\enspace}%
		\global\moditemfalse}%
}

\makeatletter
\def\@seccntformat#1{%
	\expandafter\ifx\csname c@#1\endcsname\c@section\else
	\csname the#1\endcsname\quad
	\fi}
\makeatother

\begin{document} 
	
\pagenumbering{arabic}
\pagestyle{plain}

\begin{center}
	\sc\large
	Algoritmické riešenia ťažkých problémov\\
	Domáca úloha 3
\end{center}

Autor: Marián Kravec

\section{Úloha 1 -  Random permutations}

\subsection*{a)}

Taktiku použijeme bude nasledovná, ak $a_k \geq 2^k$ tak do množiny $S$ dáme $2^{k-1}$ zamestnancov, ďalších $2^{k-1}$ do tejto množiny nedáme, a zvyšných náhodne buď do množiny dáme alebo nie. Takto vo výsledku dobe platiť $|S| \geq 2^{k-1}$ a mimo množiny ostane $a_k - |S| \geq 2^{k-1}$. ($2^{k-1} + 2^{k-1} = 2^k$)
\\

Teraz induktívne ukážeme, že pre hocijaké $k$ táto taktika je vyhrávajúca:
\\

Báza indukcie: ak $k=1$ tak $a_1 \geq 2^1 = 2$, čiže pre naše $S$ platí $|S| \geq 2^0 = 1$ a mimo $S$ ostalo $a_k - |S| \geq 2^0 = 1$, čiže bez ohľadu na to či povýšenie ostane množina $S$ alebo ostatný, aspoň jeden zamestnanec bude povýšený do predstavenstva firmy, čím vyhrávame hru.
\\

Indukčný predpoklad: hru vyhráme pre každé $n<k$
\\

Indukčný krok: chceme ukázať, že to platí aj pre $k$. Vieme, že platí $a_k \geq 2^k$ a podľa našej taktiky bude pre našu $S$ platiť $|S| \geq 2^{k-1}$, a pre zvyšok platí $a_k - |S| \geq 2^{k-1}$, čiže bez ohľadu, či bude povýšená množina $S$ alebo zvyšok povýšených bude aspoň $2^{k-1}$ zamestnancov do kariérnej tried $k-1$, čiže bude platiť $a_{k-1} \geq 2^{k-1}$ o čom ale podľa indukčného predpokladu vieme, že v takto stavu hru vyhráme, čiže hru vyhráme aj pre $k$. $\oblong$
\\

Ukázali sme, že táto taktika je víťazná pre všetky hodnoty $k$.

\subsection*{b)}

Keďže pravdepodobnosti, že predstavenstvo firmy vyberie množinu $S$ alebo ostatných je rovnaká, je irelevantné to či daný zamestnanec patrí do množiny $S$ alebo nie, vždy bude pravdepodobnosť povýšenia $\frac{1}{2}$ a pravdepodobnosť výpovede $\frac{1}{2}$. 
\\

Na to aby zamestnanec v triede $k$ sa v takomto systéme dostal do predstavenstva muselo by nastať $k$-krát, že bude povýšený (hocijaká iná postupnosť rozhodnutí by obsahovala aspoň jednu výpoveď čo by znamenalo koniec).
\\

Keďže pravdepodobnosť povýšenia je $\frac{1}{2}$ a jednotlivé rozhodnutia predstavenstva sú náhodné a nezávislé tak pravdepodobnosť, že povýšenie nastane $k$-krát je $\left(\frac{1}{2}\right)^k$. Čiže pravdepodobnosť, že zamestnanec v triede $k$ sa dostane do predstavenstva je $P(LEVEL=k)=\left(\frac{1}{2}\right)^k=\frac{1}{2^k}$.

\subsection*{c)}

Vytvorme si náhodnú premennú $X$ ktorá hovorí koľko zamestnancov sa dostane do predstavenstva. 
\\

Teraz vytvorme náhodné premenné $Y_k$ ktoré hovoria koľko zamestnancov $k$-tej triedy sa dostalo do predstavenstva. Keďže $X$ je počet cez všetky triedy a $Y_k$ pre jednotlivé triedy, vieme vzťah týchto premenných zapísať ako $X = \sum_{k=1}^{\inf} Y_k$. 
\\

Teraz si vytvorme náhodné premenné $Z_{ki}$ ktoré hovoria či $i$-ty zamestnanec $k$-tej triedy bol povýšený do predstavenstva, čiže ide o binárnu premennú kde vieme, že hodnotu $1$ (povýšený do predstavenstva) nadobudne s pravdepodobnosťou $P(LEVEL=k)=\left(\frac{1}{2}\right)^k$ (keďže je v triede $k$). Keďže $Y_k$ hovorí o počte zamestnancov v triede $k$ ktorý boli povýšený a $Z_{ki}$ o jednotlivých zamestnancoch v $k$-tej triede, vieme vzťah týchto premenných zapísať ako $Y_k = \sum_{i=1}^{a_k} Z_{ki}$.
\\

Keďže $Z_{ki}$ je binárna premenná so známou pravdepodobnosťou vieme vypočítať jej strednú hodnotu: 

$E(Z_{ki})=1*P(LEVEL=k) + 0*(1-P(LEVEL=k))=P(LEVEL=k)=\frac{1}{2^k}$.
\\

Teraz môžeme využiť linearitu strednej hodnoty a fakt, že stredná hodnota $Z_{ki}$ nezávisí od $i$ a vypočítať strednú hodnotu $Y_k$:

$E(Y_k) = E(\sum_{i=1}^{a_k} Z_{ki}) = \sum_{i=1}^{a_k} E(Z_{ki}) = \sum_{i=1}^{a_k} \frac{1}{2^k} = \frac{a_k}{2^k} $
\\

Podobne môžeme teraz využiť využiť linearitu strednej hodnoty aby sme vypočítali strednú hodnotu náhodnej premennej $X$:

$E(X) = E(\sum_{k=1}^{\inf} Y_k) = \sum_{k=1}^{\inf} E(Y_k) = \sum_{k=1}^{\inf} \frac{a_k}{2^k}$
\\

Čiže stredná (očakávaná) hodnota počtu zamestnancov povýšených do predstavenstva je:

$E(X) = \sum_{k=1}^{\inf} \frac{a_k}{2^k}$ $\oblong$
\end{document}