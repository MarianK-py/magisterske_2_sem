% !TeX spellcheck = sk_SK-Slovak
\documentclass[a4paper]{article}
\usepackage[slovak]{babel}
\usepackage[utf8]{inputenc}
\usepackage[T1]{fontenc}
\usepackage{a4wide}
\usepackage{amsmath}
\usepackage{amsfonts}
\usepackage{amssymb}
\usepackage{mathrsfs}
\usepackage[small,bf]{caption}
\usepackage{subcaption}
\usepackage{xcolor}
\usepackage{graphicx}
\usepackage{enumerate}
\usepackage{hyperref}
\usepackage{fancyvrb}
\usepackage{listings}
%\usepackage{lstautogobble}
\usepackage{stmaryrd}

\lstset{basicstyle=\ttfamily,
	mathescape=true,
	escapeinside=||%,
	%autogobble
}


\fvset{tabsize=4}


\pagestyle{empty}
\setlength{\parindent}{0pt}

\newenvironment{modenumerate}
{\enumerate\setupmodenumerate}
{\endenumerate}

\newif\ifmoditem
\newcommand{\setupmodenumerate}{%
	\global\moditemfalse
	\let\origmakelabel\makelabel
	\def\moditem##1{\global\moditemtrue\def\mesymbol{##1}\item}%
	\def\makelabel##1{%
		\origmakelabel{##1\ifmoditem\rlap{\mesymbol}\fi\enspace}%
		\global\moditemfalse}%
}

\makeatletter
\def\@seccntformat#1{%
	\expandafter\ifx\csname c@#1\endcsname\c@section\else
	\csname the#1\endcsname\quad
	\fi}
\makeatother

\begin{document} 
	
\pagenumbering{arabic}
\pagestyle{plain}

\begin{center}
	\sc\large
	Algoritmické riešenia ťažkých problémov\\
	Domáca úloha 3
\end{center}

Autor: Marián Kravec

\section{Úloha 2 - From Monte Carlo to Las Vegas}

Začnime tým, že si definujeme nový algoritmus ktorý bude fungovať tak, že vykoná obojstranný Monte Carlo algoritmus, následne skontroluje správnosť výsledku a ak tento výsledok je správny tak ho vráti inak zopakuje celý postup odznovu.
\\

Jeden krok tohto algoritmu má časovú zložitosť $T''_{\text{one step}}(n) = T(n) + T'(n)$ (keďže jediné čo spraví je, že iba spustí 2 algoritmu zo známou zložitosťou).
\\

Teraz potrebujeme zistiť aký je očakávaný počet krokov (opakovaní) nášho algoritmu. Počet krokov algoritmu označme $STEP$. Poďme postupne, pravdepodobnosť, že algoritmus skončí hneď v prvom kroku je $P(STEP=1)=1-p$ keďže to je pravdepodobnosť, že Monte Carlo algoritmus vráti správny výsledok, pravdepodobnosť, že skončí po dvoch krokoch je $P(STEP=1)=p(1-p)$ čiže v prvom kroku Monte Carlo algoritmus vráti nesprávny výsledok ($p$) a v druhom kroku správny ($1-p$). Takto by sme mohli pokračovať ďalej, vo všeobecnosti vieme povedať, že na to aby algoritmus skončil po práve $n$ krokoch musí najprv $n-1$-krát vygenerovať nesprávne riešenie a nakoniec správne čo má pravdepodobnosť $P(STEP=n)=p^{n-1}(p-1)$.
\\

Teraz chceme vypočítať stredný počet krokov:

\begin{align*}
	E(STEP) &= \sum_{i=1}^{\inf} iP(STEP=i) =\\
	&= \sum_{i=1}^{\inf} ip^{i-1}(p-1) =\\
	&= (1-p)\sum_{i=1}^{\inf} ip^{i-1} =\\
	&\text{Teraz si pomôžeme trochu internetom: \href{https://math.stackexchange.com/questions/1000866/the-summation-of-sequences-n-alphan#1000872}{ZDROJ}}\\
	&= (1-p)\frac{1}{(1-p)^2} =\\
	&= \frac{1}{1-p}
\end{align*}

Vieme, že očakávaný čas algoritmu bude očakávaný počet krokov algoritmu vynásobený časovou zložitosťou jedného kroku čiže:

\begin{align*}
	E(T''(n)) &= E(STEP)T''_{\text{one step}}(n) =\\
	 &= \frac{1}{1-p}(T(n) + T'(n)) =\\
	 &= \frac{T(n) + T'(n)}{1-p} 
\end{align*}

Čiže očakávaná čas tohto algoritmu je $\frac{T(n) + T'(n)}{1-p}$. Keďže tento algoritmu vždy vráti správnu hodnotu a jeho čas zaleží od náhodných čísel (použitých v Monte Carlo podalgoritme) ide o Las Vegas s požadovaným očakávaným časom. $\oblong$ 



\end{document}