% !TeX spellcheck = sk_SK-Slovak
\documentclass[a4paper]{article}
\usepackage[slovak]{babel}
\usepackage[utf8]{inputenc}
\usepackage[T1]{fontenc}
\usepackage{a4wide}
\usepackage{amsmath}
\usepackage{amsfonts}
\usepackage{amssymb}
\usepackage{mathrsfs}
\usepackage[small,bf]{caption}
\usepackage{subcaption}
\usepackage{xcolor}
\usepackage{graphicx}
\usepackage{enumerate}
\usepackage{hyperref}



\pagestyle{empty}
\setlength{\parindent}{0pt}

\newenvironment{modenumerate}
{\enumerate\setupmodenumerate}
{\endenumerate}

\newif\ifmoditem
\newcommand{\setupmodenumerate}{%
	\global\moditemfalse
	\let\origmakelabel\makelabel
	\def\moditem##1{\global\moditemtrue\def\mesymbol{##1}\item}%
	\def\makelabel##1{%
		\origmakelabel{##1\ifmoditem\rlap{\mesymbol}\fi\enspace}%
		\global\moditemfalse}%
}

\makeatletter
\def\@seccntformat#1{%
	\expandafter\ifx\csname c@#1\endcsname\c@section\else
	\csname the#1\endcsname\quad
	\fi}
\makeatother

\begin{document} 
	
\pagenumbering{arabic}
\pagestyle{plain}

\begin{center}
	\sc\large
	Umelá inteligencia - cvičenia 10
\end{center}

Autor: Marián Kravec

\section{Úloha 2}

Budeme modelovať čudný semafor, pričom to čo pozorujeme, je či autá prechádzajú cez križovatku.
\\

Náš model bude mať 3 skryté stavy: červená (č), oranžová (o) a zelená (z).

Našou pozorovaná premenná bude mať iba 2 hodnoty: autá idú (ai), autá stoja (as).
\\

Inicialne začne vždy kvôli bezpečnosti cestnej premávky v stave červená.
\\ 

Tranzičná matica bude nasledovná:

\begin{table}[h!]
	\begin{tabular}{|l|l|l|l|}
		\hline
		$s_i$ \char`\\ $s_{i+1}$ & č   & o   & z   \\ \hline
		č                    & 0.7 & 0.3 & 0   \\ \hline
		o                    & 0.3 & 0.4 & 0.3 \\ \hline
		z                    & 0   & 0.2 & 0.8 \\ \hline
	\end{tabular}
\end{table}

Čiže s červenej nikdy neprejde hneď na zelenú (a naopak). Ale z oranžovej sa môže vrátiť aj späť do stavu z ktoré ho sa do oranžovej dostal (zvláštnosť tohto semaforu).
\\

Emisná matica je nasledovná:

\begin{table}[h!]
	\begin{tabular}{|l|l|l|}
		\hline
		$s_i$ \char`\\ $v_i$ & as   & ai     \\ \hline
		č                    & 0.95 & 0.05  \\ \hline
		o                    & 0.3  & 0.7  \\ \hline
		z                    & 0.01 & 0.99  \\ \hline
	\end{tabular}
\end{table}

Na červenú ide len pár odvážlivcov, na oranžovú už viac ako polovica a na zelenú takmer všetci.
\\

Po vygenerovaní 10000 krokov modelu, sme získali pozorovania ktoré majú skóre: $-5189.26$

\end{document}