% !TeX spellcheck = sk_SK-Slovak
\documentclass[a4paper]{article}
\usepackage[slovak]{babel}
\usepackage[utf8]{inputenc}
\usepackage[T1]{fontenc}
\usepackage{a4wide}
\usepackage{amsmath}
\usepackage{amsfonts}
\usepackage{amssymb}
\usepackage{mathrsfs}
\usepackage[small,bf]{caption}
\usepackage{subcaption}
\usepackage{xcolor}
\usepackage{graphicx}
\usepackage{enumerate}
\usepackage{hyperref}



\pagestyle{empty}
\setlength{\parindent}{0pt}

\newenvironment{modenumerate}
{\enumerate\setupmodenumerate}
{\endenumerate}

\newif\ifmoditem
\newcommand{\setupmodenumerate}{%
	\global\moditemfalse
	\let\origmakelabel\makelabel
	\def\moditem##1{\global\moditemtrue\def\mesymbol{##1}\item}%
	\def\makelabel##1{%
		\origmakelabel{##1\ifmoditem\rlap{\mesymbol}\fi\enspace}%
		\global\moditemfalse}%
}

\makeatletter
\def\@seccntformat#1{%
	\expandafter\ifx\csname c@#1\endcsname\c@section\else
	\csname the#1\endcsname\quad
	\fi}
\makeatother

\begin{document} 
	
\pagenumbering{arabic}
\pagestyle{plain}

\begin{center}
	\sc\large
	Umelá inteligencia - cvičenia 2
\end{center}

Autor: Marián Kravec

\section{Úloha 1}

\begin{table}[!h]
\begin{tabular}{|p{0.15\textwidth}|p{0.10\textwidth}|p{0.12\textwidth}|p{0.5\textwidth}|}
	\hline
	Algoritmus                 & Časová zložitosť & Priestorová zložitosť & Pár viet \\ \hline
	Breadth-First Search (BFS) & $O(b^d)$         & $O(b^d)$              & ($d$ - hĺbka najplytkejšieho cieľa)Algoritmus prechádza strom po úrovniach. Jeho nevýhodou je pomerne veľké množstvo pamäte keďže si musí pamätať celé úrovne ktorých veľkosť môže rásť exponenciálne     \\ \hline
	Depth-First Search (DFS)   & $O(b^m)$         & $O(bm)$               & ($m$ - hĺbka stromu) Algoritmus prechádza strom do hĺbky, čiže kým sa z daného vrcholu dá pokračovať nižšie tak tam pokračuje. Nevýhodou je, že pri hlbokých stromoch môže tento algoritmus tráviť veľké množstvo čase hlboko v strome aj napriek tomu, že možné riešenie je vo vrchných vrstvách inej vetvy \\ \hline
	Depth-Limited Search (DLS) & $O(b^l)$         & $O(bl)$ & Modifikácia DFS algoritmu, kde má algoritmus povedanú maximálnu hĺbku $l$ do ktorej sa má vnárať. Výhodou je lepší čas ako DFS avšak nevýhodou je, že nájdenie riešenia nie je zaručené (a to ani v prípade, že vieme, že niekde v strome riešenie je) \\ \hline
	Iterative Deepening Depth-First Search (IDDFS) & $O(b^d)$         & $O(bd)$               & Modifikácia DLS algoritmu kde sa hodnota limitu $l$ postupne zvyšuje, kým algoritmus nenájde riešenie. Týmto rieši problém DFS, že sa môže dlho pohybovať na nevhodnej vetve (vďaka DLS) a zároveň rieši problém DLS, že nemusí nájsť riešenie postupným zväčšovaním hĺbky \\ \hline
	Uniform Cost Search (UCS) & $O(b^{1+\lfloor l/e \rfloor}))$ & $O(b^{1+\lfloor l/e \rfloor})$ & ($l$ - dĺžka najkratšej cesty, $e$ - minimálna cena hrany) V tomto algoritme vkladáme nové vrcholy do prioritnej fronty na základe ceny. Ide o variant Dijkstrovho algoritmu \\ \hline
	Best-First Search (BeFS) & $O(b^m)$ & $O(b^m)$ & Tento algoritmus na základe heuristiky určí pre každý nový očakávanú cenu do cieľa a následne podobne ako UCS ho vložíme do prioritnej fronty z ktorej následne vyberáme, ďalší prehľadávaný vrchol \\ \hline
	A* & $O(b^d)$ & $O(b^d)$ & Ide o rozšírenie BeFS (v podstate pridanie UCS) kde keď nové vrcholy pridávame do prioritnej fronty na základe súčty ich doterajšej ceny a ceny ktorú predpokladá heuristika do cieľa \\ \hline
\end{tabular}
\end{table}

\end{document}