% !TeX spellcheck = sk_SK-Slovak
\documentclass[a4paper]{article}
\usepackage[slovak]{babel}
\usepackage[utf8]{inputenc}
\usepackage[T1]{fontenc}
\usepackage{a4wide}
\usepackage{amsmath}
\usepackage{amsfonts}
\usepackage{amssymb}
\usepackage{mathrsfs}
\usepackage[small,bf]{caption}
\usepackage{subcaption}
\usepackage{xcolor}
\usepackage{graphicx}
\usepackage{enumerate}
\usepackage{hyperref}
\usepackage{fancyvrb}
\usepackage{listings}
%\usepackage{lstautogobble}
\usepackage{stmaryrd}

\newsavebox\MBox
\newcommand\Cline[2][red]{{\sbox\MBox{$#2$}%
		\rlap{\usebox\MBox}\color{#1}\rule[-1.2\dp\MBox]{\wd\MBox}{0.5pt}}}

\lstset{basicstyle=\ttfamily,
	mathescape=true,
	escapeinside=||%,
	%autogobble
}


\fvset{tabsize=4}


\pagestyle{empty}
\setlength{\parindent}{0pt}

\newenvironment{modenumerate}
{\enumerate\setupmodenumerate}
{\endenumerate}

\newif\ifmoditem
\newcommand{\setupmodenumerate}{%
	\global\moditemfalse
	\let\origmakelabel\makelabel
	\def\moditem##1{\global\moditemtrue\def\mesymbol{##1}\item}%
	\def\makelabel##1{%
		\origmakelabel{##1\ifmoditem\rlap{\mesymbol}\fi\enspace}%
		\global\moditemfalse}%
}

\makeatletter
\def\@seccntformat#1{%
	\expandafter\ifx\csname c@#1\endcsname\c@section\else
	\csname the#1\endcsname\quad
	\fi}
\makeatother

\begin{document} 
	
\pagenumbering{arabic}
\pagestyle{plain}

\begin{center}
	\sc\large
	Programovanie paralélnych a distribuovaných systémov\\
	Domáca úloha 9
\end{center}

Autor: Marián Kravec

\section{Úloha 1}

Algoritmus ktorý napĺňa zoznam $z$ prvkami zo zoznamu $x$ je následovný:

\begin{enumerate}
	\item pridaj na koniec pomocou príkazu $z[N-1] := u$ a zároveň vymeň všetky takéto dvojice $z[N-1-2*k], z[N-2*k] := z[N-2*k], z[N-1-2*k], \lfloor \frac{N}{2} \rfloor +1 \leq k>0$ (v podstate ide o to, že chceme vymeniť predposledný s predpredposledným, potom dvojicu pred nimi a tak ďalej)
	\item teraz vymeň dvojice $z[N-2*k], z[N+1-2*k] := z[N-2*], z[N-1-2*k], \lfloor \frac{N}{2} \rfloor +1 \leq k > 0$ (tu vlastne vymieňame dvojice ktoré sme nevymenili v predchádzajúcom kroku, čiže posledný s predposledným, dvojice pred nimi a tak ďalej) 
\end{enumerate} 

(v oboch prípadoch som definovával vymieňané dvojice od konca zoznamu keďže na koniec pridávame a záleží na tom, či, posledný prvok smieme vymieňať alebo nie (keďže v prvom kroku ho vymieňať nemôžeme keďže tá hodnota je už použitá v inom príkaze))

Ukážme si to na príklade zoznamu dĺžky 5:

Na začiatku majme zoznam $z$ vyzerajúci takto:

$\_ \quad \_ \quad \_ \quad \_ \quad \_$

A chceme tam dostať zoznam $x$ vyzerajúci takto:

$a \quad b \quad c \quad d \quad e $

Dvojice ktoré sme vymenili označíme červeným podčiarknutím a do zátvoriek si zapíšeme ktorý z dvoch krokov vykonávame a koľký krok celkovo to je.

Teraz poďme na algoritmus:

$\Cline[red]{\_ \quad \_} \quad \Cline[red]{\_ \quad \_} \quad a$ (1, 1)

$\_ \quad \Cline[red]{\_ \quad \_} \quad \Cline[red]{a \quad \_}$ (2, 2)

$\Cline[red]{\_ \quad \_} \quad \Cline[red]{a \quad \_} \quad b$ (1, 3)

$\_ \quad \Cline[red]{a \quad \_} \quad \Cline[red]{b \quad \_}$ (2, 4)

$\Cline[red]{a \quad \_} \quad \Cline[red]{b \quad \_} \quad c$ (1, 5)

$a \quad \Cline[red]{b \quad \_} \quad \Cline[red]{c \quad \_}$ (2, 6)

$\Cline[red]{b \quad a} \quad \Cline[red]{c \quad \_} \quad d$ (1, 7)

$b \quad \Cline[red]{c \quad a} \quad \Cline[red]{d \quad \_}$ (2, 8)

$\Cline[red]{c \quad b} \quad \Cline[red]{d \quad a} \quad e$ (1, 9)

$c \quad \Cline[red]{d \quad b} \quad \Cline[red]{e \quad a}$ (2, 10)

Takže po 10 krokoch (čo je presne 2N operácii) máme v zozname $z$ zoznam $x$ (len trochu poprehadzovaný) (v podstate nám stačí 9 ale pre všeobecné N to vyjde 2N-1 čo je zanedbateľné):

$c \quad d \quad b \quad e \quad a$
\\

Tento algoritmus funguje vďaka tomu, že efektívne posúva prvky z konca zoznamu na začiatok a naopak vďaka čomu sa na koniec dostávajú postupne všetky prvky z pôvodného zoznamu a môže byť nahradené.
\newpage

\section{Úloha 2}

V prípade, že môžeme pridávať prvky aj na začiatok, môžeme využiť modifikáciu algoritmu z úlohy 1 a využiť, že tento algoritmus nie len efektívne presúval prvky z konca na začiatok ale aj naopak.
\\

Akurát musíme dávať pozor, v ktorom kroku nášho algoritmu môžeme pridať príkaz na pridávanie na začiatok zoznamu keďže nechceme aby ten istý prvok bol v dvoch príkazoch. 

Ak máme párny počet prvkov tak pridáme pridávanie na začiatok zoznamu do 2. kroku algoritmu (v 1. kroku sa prvý prvok vymieňa s druhým), ak je počet prvkov nepárny tak pridávame na začiatok v 1. kroku algoritmus (zase v 2. sa mení prvý s druhým).
\\

Poďme si to ukázať na dvoch príkladoch a to dĺžky 5 a 6 (keďže parita počtu je dôležitá):

Najskôr dĺžky 5 (zoznamy sú totožné z úlohy 1):

$\Cline[red]{\_ \quad \_} \quad \Cline[red]{\_ \quad \_} \quad a$ (1, 1)

$b \quad \Cline[red]{\_ \quad \_} \quad \Cline[red]{a \quad \_}$ (2, 2)

$\Cline[red]{\_ \quad b} \quad \Cline[red]{a \quad \_} \quad c$ (1, 3)

$d \quad \Cline[red]{a \quad b} \quad \Cline[red]{c \quad \_}$ (2, 4)

$\Cline[red]{a \quad d} \quad \Cline[red]{c \quad b} \quad e$ (1, 5)
\\

Teraz pre dĺžku 6 (iba na koniec $x$ pridáme f):

$b \quad \Cline[red]{\_ \quad \_} \quad \Cline[red]{\_ \quad \_} \quad a$ (1, 1)

$\Cline[red]{\_ \quad b} \quad \Cline[red]{\_ \quad \_} \quad \Cline[red]{a \quad \_}$ (2, 2)

$d \quad \Cline[red]{\_ \quad b} \quad \Cline[red]{a \quad \_} \quad c$ (1, 3)

$\Cline[red]{\_ \quad d} \quad \Cline[red]{a \quad b} \quad \Cline[red]{c \quad \_}$ (2, 4)

$f \quad \Cline[red]{a \quad d} \quad \Cline[red]{c \quad b} \quad e$ (1, 5)
 
$\Cline[red]{a \quad f} \quad \Cline[red]{a \quad d} \quad \Cline[red]{e \quad b}$ (2, 6)
\\

Vidíme, že keďže pri nepárnom počte prvkov pridávame prvok v každom kroku tak nám stačí N operácii. Podobne v prípade párneho počtu pridávame 2 prvky za 2 kroky takže nám takisto stačí N operácii.
\\

Takže tento modifikovaný algoritmu dokáže prekopírovať $x$ do $z$ na N operácii.
\end{document}