% !TeX spellcheck = sk_SK-Slovak
\documentclass[a4paper]{article}
\usepackage[slovak]{babel}
\usepackage[utf8]{inputenc}
\usepackage[T1]{fontenc}
\usepackage{a4wide}
\usepackage{amsmath}
\usepackage{amsfonts}
\usepackage{amssymb}
\usepackage{mathrsfs}
\usepackage[small,bf]{caption}
\usepackage{subcaption}
\usepackage{xcolor}
\usepackage{graphicx}
\usepackage{enumerate}
\usepackage{hyperref}
\usepackage{fancyvrb}
\usepackage{listings}
%\usepackage{lstautogobble}
\usepackage{stmaryrd}

\lstset{basicstyle=\ttfamily,
	mathescape=true,
	escapeinside=||%,
	%autogobble
}


\fvset{tabsize=4}


\pagestyle{empty}
\setlength{\parindent}{0pt}

\newenvironment{modenumerate}
{\enumerate\setupmodenumerate}
{\endenumerate}

\newif\ifmoditem
\newcommand{\setupmodenumerate}{%
	\global\moditemfalse
	\let\origmakelabel\makelabel
	\def\moditem##1{\global\moditemtrue\def\mesymbol{##1}\item}%
	\def\makelabel##1{%
		\origmakelabel{##1\ifmoditem\rlap{\mesymbol}\fi\enspace}%
		\global\moditemfalse}%
}

\makeatletter
\def\@seccntformat#1{%
	\expandafter\ifx\csname c@#1\endcsname\c@section\else
	\csname the#1\endcsname\quad
	\fi}
\makeatother

\begin{document} 
	
\pagenumbering{arabic}
\pagestyle{plain}

\begin{center}
	\sc\large
	Programovanie paralélnych a distribuovaných systémov\\
	Domáca úloha 8
\end{center}

Autor: Marián Kravec

\section{Úloha 1}

Tento program bude fungovať. Dokážeme si to tým, že dokážeme správnosť dvoch prípadov:

\begin{itemize}
	\item generál je spoľahlivý -> v takomto prípade, obe spoľahlivé procesy (bez generála a jedného nespoľahlivého) dostanú jednu pravdivú informáciu od generála a druhú totožnú od druhého spoľahlivého, preto bez ohľadu akú informáciu dostane od posledného (nespoľahlivého) procesu tá už nezmení nič na tom, že dostal dve správne informácie ktoré tvoria väčšinu
	\item generál je nespoľahlivý -> tento prípad si rozdelíme na dve podprípady (keďže len tie existujú):
	\begin{itemize}
		\item generál povedal všetkým to isté -> keďže všetky ostatné procesy sú spoľahlivé navzájom si informáciu od generála pravdivo vďaka čomu budú mať všetky spoľahlivé procesy tri-krát tú istú informáciu takže sa rozhodnú rovnako (keďže generál je nespoľahlivý konkrétna hodnota nás netrápi)
		\item generál dvom povie A a tretiemu B -> obaja čo dostanú A dostanú ešte jednu informáciu A od druhého preto sa rozhodnú pre A, tretí čo dostane B zistí, že obaja zvyšný dostali A preto si vyberie majoritu čiže A, čiže sa všetci traja spoľahlivý zhodnú
	\end{itemize}
\end{itemize}

Ukázali sme, že vo všetkých prípadoch sa všetci spoľahlivý zhodnú na rovnakej odpovedi, čo znamená, že algoritmu funguje správne.
\\

Naneštastie tento algoritmus nefunguje pre všeobecné $n$. Ukážeme si to už na jednoduchom prípade $n=5$ s jedným nespoľahlivým.

Uvažujme, že nespoľahlivý je generál a dvom povie A a dvom B, v takomto prípade všetci spoľahlivý zistia informáciu dva-krát A a dva-krát B, z toho vyplýva, že sa nevedia rozhodnúť na základe väčšinového princípu. 
\\

Ak algoritmus modifikujeme tak, že ak sa nevie proces rozhodnúť vráti 0 mal by algoritmus fungovať pre všeobecné $n$ z toho jeden nespoľahlivý. Keďže ak by bol generál spoľahlivý tak $n-2$ spoľahlivých procesov prebije jeden nespoľahlivý a ak je generál nespoľahlivý tak bud dá väčšine rovnakú informáciu a keďže ju všetky spoľahlivo pošlú ostatným alebo dá rôznu informáciu práve polovici a vtedy sa zhodnú na 0.

\end{document}