% !TeX spellcheck = sk_SK-Slovak
\documentclass[a4paper]{article}
\usepackage[slovak]{babel}
\usepackage[utf8]{inputenc}
\usepackage[T1]{fontenc}
\usepackage{a4wide}
\usepackage{amsmath}
\usepackage{amsfonts}
\usepackage{amssymb}
\usepackage{mathrsfs}
\usepackage[small,bf]{caption}
\usepackage{subcaption}
\usepackage{xcolor}
\usepackage{graphicx}
\usepackage{enumerate}
\usepackage{hyperref}
\usepackage{fancyvrb}
\usepackage{listings}
%\usepackage{lstautogobble}
\usepackage{stmaryrd}

\lstset{basicstyle=\ttfamily,
	mathescape=true,
	escapeinside=||%,
	%autogobble
}


\fvset{tabsize=4}


\pagestyle{empty}
\setlength{\parindent}{0pt}

\newenvironment{modenumerate}
{\enumerate\setupmodenumerate}
{\endenumerate}

\newif\ifmoditem
\newcommand{\setupmodenumerate}{%
	\global\moditemfalse
	\let\origmakelabel\makelabel
	\def\moditem##1{\global\moditemtrue\def\mesymbol{##1}\item}%
	\def\makelabel##1{%
		\origmakelabel{##1\ifmoditem\rlap{\mesymbol}\fi\enspace}%
		\global\moditemfalse}%
}

\makeatletter
\def\@seccntformat#1{%
	\expandafter\ifx\csname c@#1\endcsname\c@section\else
	\csname the#1\endcsname\quad
	\fi}
\makeatother

\begin{document} 
	
	\pagenumbering{arabic}
	\pagestyle{plain}
	
	\begin{center}
		\sc\large
		Neurónové siete\\
		Projekt 2\\
		Self-organizing map
	\end{center}
	
	Autor: Marián Kravec
	\\
	
	\section{Úvod}
	
	V tejto úlohe sa snažíme natrénovať dvojrozmernú štvoruholníkovú SOM na vizualizáciu 8 rozmerných dát (z toho jedna kategória). Ide o dataset seeds z UCI Machine Learning Repository.
	
	\section{Dáta}
	
	Máme dataset tvorený 210 dátovými bodmi ktoré majú 8 rozmerov, 7 rozmerov sú parametre bodu a ôsmi je kategória. Tento dataset rozdelíme v pomere 5:2 na trénovacie a testovacie dáta. Takto získame 150 bodov na trénovanie a 60 bodov na testovanie.
	
	\section{Architektúra a hyperparametre}
	
	Pri výbere modelu sme skúšali tri rôzne normy pre vzdialenosti na sieti, konkrétne sme skúsili normy $L_1$, $L_2$, $L_{max}$. Zároveň sme skúsili trénovať s diskrétnou aj spojitou funkciou susednosti. V neposlednej rade sme skúšali aj viac štartovacích hodnôt parametra $\alpha$, konkrétne sme skúsili hodnoty: $\{0.5, 0.7, 1, 2, 5, 10\}$ (finálna hodnota $\alpha$ bola pre všetky modely $0.01$)
	\\
	
	Všetky modely boli trénované na 500 epoch. Veľkosti všetkých trénovaných sietí boli $10x10$. Parameter $\lambda$ začína na hodnote priemeru rozmerov siete ($\frac{\#rows + \#columns}{2}$) aby sa na začiatku zmena propagovala po celej sieti a končí na hodnote 1 aby ku koncu bola zmena iba lokálna. 
	\\
	
	Nakoniec sa ako najlepší model ukázal model využívajúci normu $L_1$, diskrétnu funkciu susednosti a počiatočný parameter $\alpha=1$.
	
	\section{Výsledky modelu}
	
	Výsledný model sme trénovali na 1000 epoch s parametrami najlepšieho modelu z predchádzajúcej časti.

	\begin{figure}[!h]
		\centering
		\includegraphics[width=\textwidth]{../errors.png}
		\caption{Quantizačnej chyby a priemernej zmeny pozície neurónu počas trénovania modelu }
	\end{figure}
	
\end{document}